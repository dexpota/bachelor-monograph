% La suddivisione in parti non � necessaria nella monografia \part{Nome della parte}
\chapter{Introduzione}
\section{Seac02}
La Seac02 s.r.l. trova sede a Torino in Via Avogadro 4 incubata dal politecnico di Torino � nata nel gennaio 2003.  Azienda che in tre anni ha saputo fare il salto di qualit�, passando da una dimensione di start-up innovativa a quella di giovane impresa dinamica e affermata nel settore delle soluzioni per la comunicazione e in particolare nell'uso della realt� aumentata come strumento di interazione. 

Alla forte competenza tecnologica e al management aziendale di Seac02 si sono affiancate le azioni di accompagnamento strategico e manageriale dell'incubatore, le collaborazioni con Torino Wireless e il Politecnico di Torino, il supporto finanziario dei fondi di Venture Capital del Distretto come Piemontech che hanno consentito all'azienda di sviluppare soluzioni vincenti.

\subsection{Obiettivo}

La Seac02 ha come obiettivo quello di fornire al mercato del software sistemi di visualizzazione tridimensionale mirati al miglioramento della qualit� e dell'efficienza dei processi d'ingegneria, marketing, vendita e comunicazione.  Tra i vari prodotti � possibile trovare applicativi per la progettazione assistita dall'elaboratore (strumenti CAD), per la realt� aumentata e la realt� virtuale. Quest'ultime sono sicuramente il punto di forza dell'azienda, che con la sua innovazione punta soprattutto a rinnovare il modo di interagire delle aziende con i propri clienti.

\subsection{Prodotti}

Tra i principali prodotti della Seac02 possiamo trovare LinceoVR, Eligo, Display designer e Virtual rooms. LinceoVR � un applicativo per la realt� aumentata, il rendering in tempo reale e l'animazione. Inoltre � compatibile con il robot Wowee Rovio e gli occhialini Vuzix iWear con camera per la realt� aumentata. LinceoVR � una vera innovazione per la maggior parte degli impegnativi bisogni sulla visuale in tempo reale del marketing. Supporta hardware ATI NVIDIA e gira su openGL e su windows. Offre una serie di plugin per le diverse esigenze, � alla portata di tutti.

Eligo SDK � una soluzione multi piattaforma per le applicazioni di realt� virtuale e aumentata. Permette di controllare ogni tecnologia di tracciamento per la realt� aumentata presente sul mercato e  di ottenere i migliori dettagli con il rendering interattivo in tempo reale, grazie a un ricco set di semplici API. Eligo SDK permette l'integrazione di queste tecnologie su diverse piattaforme web, desktop e mobile.

Le soluzioni Seac02 di realt� aumentata comprendono anche una serie di prodotti per l'interazione avanzata con la realt� virtuale, questi vanno sotto il nome di Virtual Rooms.  Queste tecnologie sono pensate per aiutare la verifica di modelli virtuali, attraverso un riscontro visuale. Il loro obiettivo � quello di aiutare il processo decisionale, ridurre considerevolmente il tempo dedicato al design, ottimizzare le fasi di produzione, supportare la formazione di operatori, migliorare il modo in cui le aziende sono percepite e offrire ai clienti e agli operatori un legame emotivo con il prodotto, incrementando le opportunit� di business.

\section{Motivazioni della monografia}

Lo stage ha come obiettivo quello di fornire un esperienza pratica allo studente. Costituisce un'occasione per il l'inserimento nel mondo produttivo al fine di stabilire un primo contatto ed, al contempo, di svolgere un periodo di addestramento pratico. Lo studente avr� quindi l'occasione di mettere in pratica le proprie conoscenze e di imparare a inserirsi in un gruppo di lavoro estraneo.

La monografia ha come obiettivo quello di verificare ci� che lo studente ha affrontato durante lo stage. Lo scopo � quello di permettere allo studente di realizzare un vero e proprio progetto che sia basato sull'analisi delle problematiche. La monografia conterr� quindi tutti i dettagli sullo sviluppo del lavoro assegnato. La motivazione finale della monografia � quindi quella di permettere una valutazione oggettiva del lavoro svolto in azienda.
