\chapter*{Sommario}
\addcontentsline{toc}{chapter}{Sommario}	%In questo modo compare nell'indice ma senza numerazione
Lo scopo di questa monografia � di esporre in maniera chiara e concisa l'attivit� svolta all'interno dell'azienda Seac02.  Per aiutare il lettore nella consultazione, la monografia � stata suddivisa in tre parti principali la prima che descrive l'azienda e le attivit� svolte, la seconda in cui si parla dell'analisi e della fase progettuale e la terza in cui si affrontano i dettagli di realizzazione dell'applicativo. Un'ultima parte � dedicata al commento sul lavoro svolto e uno sguardo critico al risultato ottenuto, con un occhio ai possibili sviluppi futuri.

L'argomento su cui verte la monografia � la realizzazione di un applicativo per il calcolo dell'ambient occlusion e la creazione della texture associata. L'ambient occlusion � un metodo di ombreggiatura utilizzato nella grafica computerizzata per conferire realismo ai modelli di riflessione locale. La texture � un'immagine bidimensionale utilizzata per rivestire la superficie di un oggetto virtuale, � utilizzata per aggiungere dettagli al modello. Nel nostro caso la texture � utilizzata per rappresentare l'ombreggiatura locale del modello.

Questo applicativo dovr� agire su un gran numero di modelli in maniera automatica, senza nessun tipo d'intervento umano. Dovr� quindi essere in grado di determinare una parametrizzazione valida per il modello, di campionare l'occlusione sull'intera superficie virtuale e scrivere il risultato su una texture.

L'obiettivo � quello di risolvere alcune problematiche presenti nel lavoro fin qui svolto. Queste riguardavano due ambiti, il primo verte sulla resa grafica della texture mentre il secondo sull'individuazione dei punti di campionamento sulla texture. La prima parte della monografia � dedicata all'analisi di questi problemi e alle conclusioni tratte per la loro risoluzione.

Ci siamo posti l'obiettivo di ottenere gradualmente quello che ci siamo prefissati. Siamo partiti dallo scomporre il problema in parti pi� piccole e semplici da gestire. Per ciascuna di esse abbiamo analizzato le varie possibilit�, e di conseguenza scelto il miglior modo per raggiungere lo scopo nei tempi stabiliti dallo stage. 

Riteniamo che non tutte le parti del software siano ancora adesso perfette. Alcune parti soffrono ancora di problemi. Crediamo anche che l'applicativo si possa migliorare, soprattutto la parte sul calcolo della ambient occlusion e il calcolo della texture. Queste questioni sono trattate nell'ultima parte della monografia, in cui si analizzano i problemi riscontrati e si propongono varie soluzioni. Inoltre sono consigliate varie strategie che si potrebbero adottare per migliorare le prestazioni dell'applicativo, in particolare si trattano le possibilit� di parallelizzazione dell'algoritmo.
